\documentclass[UTF8,a4paper]{ctexart}
\usepackage{listings}
\usepackage{amsmath}
\usepackage{amssymb}
\usepackage{color, soul}
\usepackage{graphicx}
\title{数据分析作业(二)}
\author{王福东}
\date{\today}
\begin{document}
\maketitle
\begin{enumerate}
\item Answer 1 \\
{\color{yellow} \large \textbf{$Assume:$}}\\
\begin{center}
\emph{$H_0$}:\emph{M}=10,\,\,V.S.\,\,\,\emph{$H_1$}:\emph{M}$\neq$ 10
\end{center}
{\color{blue} \large \textbf{$R\,\,\, Command:$}}\\

\begin{lstlisting}[language=R]
> w=c(9.8,10.1,9.7,9.9,10,10,9.8,9.7,9.8,9.9)
> binom.test(sum(w>10),sum(w>10)+sum(w<10),al="two.sided")
\end{lstlisting}
{\color{red} \large \textbf{$ Result:$}}\\

\begin{verbatim}
	Exact binomial test

data:  sum(w > 10) and sum(w > 10) + sum(w < 10)
number of successes = 1, number of trials = 8, p-value = 0.07031
alternative hypothesis: true probability of success is not equal to 0.5
95 percent confidence interval:
 0.003159724 0.526509671
sample estimates:
probability of success 
                 0.125 

\end{verbatim}

\textcolor[rgb]{0.28,0.9,0.8} {\large \textbf{$ Conclusion:$}}\\
$p-value$大于显著性水平(默认$\alpha =0.05$),可以接受原假设,也就是说不需要对控制做调整

\item Answer\,2
{\color{yellow} \large \textbf{$Assume:$}}\\
\begin{center}
\emph{$H_0$}:\emph{M}=32.32,\,\,V.S.\,\,\,\emph{$H_1$}:\emph{M}$\neq$ 32.32
\end{center}
{\color{blue} \large \textbf{$R\,\,\, Command:$}}\\
\begin{lstlisting}[language=R]
> w<-c(432,82.46,46.19,43.14,21.59,12.61,23.61,22.2,11.31,
105.53,83.57,21.77,11.95,42.17,6.30,39.03)
> binom.test(sum(w>32.32),sum(w>32.32)+sum(w<32.32),
al="greater")
\end{lstlisting}
{\color{red} \large \textbf{$ Result:$}}\\
\begin{verbatim}
	Exact binomial test

data:  sum(w > 32.32) and sum(w > 32.32) + sum(w < 32.32)
number of successes = 8, number of trials = 16, p-value = 0.5982
alternative hypothesis: true probability of success is less than 0.5
95 percent confidence interval:
 0.0000000 0.7213973
sample estimates:
probability of success 
                   0.5 
\end{verbatim}
\textcolor[rgb]{0.28,0.9,0.8} {\large \textbf{$ Conclusion:$}}\\
$p-value$大于显著性水平(默认$\alpha =0.05$),可以接受原假设,也即2003年较2002年互联网上网人数没有明显增加

\item Answer\,3
{\color{yellow} \large \textbf{$Assume:$}}\\
\begin{center}
\emph{$H_0$}:无趋势,\,\,V.S.\,\,\,\emph{$H_1$}:有单调递减的趋势
\end{center}
{\color{blue} \large \textbf{$R\,\,\, Command:$}}\\
\begin{lstlisting}[language=R]
library(tseries)
w<-c(2.3,2.396,2.1,2.442,1.81,2.007,1.766,1.224,1.411,
1.438,0.892,1.167)
D<-numeric()
if(length(w)%%2==0){
  c=length(w)/2
  for(i in 1:c)
    D[i]=w[i]-w[i+c]}
if(length(w)%%2==1){
  c=length(w)/2+0.5
  for(i in 1:c-1)
    D[i]=w[i]-w[i+c]}
binom.test(sum(D>0),sum(D>0)+sum(D<0),al="two.sided")
\end{lstlisting}
{\color{red} \large \textbf{$ Result:$}}\\
\begin{verbatim}
	Exact binomial test

data:  sum(D > 0) and sum(D > 0) + sum(D < 0)
number of successes = 6, number of trials = 6, p-value = 0.03125
alternative hypothesis: true probability of success is not equal to 0.5
95 percent confidence interval:
 0.5407419 1.0000000
sample estimates:
probability of success 
                     1 
\end{verbatim}
\textcolor[rgb]{0.28,0.9,0.8} {\large \textbf{$ Conclusion:$}}\\
$p-value$小于显著性水平,不接受原假设,可以认为数据有单调下降的趋势

\item Answer\,4
{\color{yellow} \large \textbf{$Assume:$}}\\
\emph{$H_0$}:是随机的,\,\,V.S.\,\,\,\emph{$H_1$}:不是随机的\\

{\color{blue} \large \textbf{$R\,\,\, Command:$}}\\

\begin{lstlisting}[language=R]
w<-c(3.6,3.9,4.1,3.6,3.8,3.7,3.4,4.0,3.8,4.1,3.9,
4.0,3.8,4.2,4.1)
  x<-numeric()
  for(i in 1:length(w)){
    if(w[i]-median(w)>0)
      x[i]=1
    if(w[i]-median(w)<=0)
      x[i]=0
  }
  x<-factor(x)
  runs.test(x)
\end{lstlisting}
{\color{red} \large \textbf{$ Result:$}}\\

\begin{verbatim}
Runs Test

data:  x
Standard Normal = 1.008, p-value = 0.3134
alternative hypothesis: two.sided
\end{verbatim}

\textcolor[rgb]{0.28,0.9,0.8} {\large \textbf{$ Conclusion:$}}\\
$p-value$











\end{enumerate}
\end{document}
